\documentclass[11pt]{amsart}
\usepackage{amsmath,amsfonts,amssymb,amsthm,verbatim,multirow,url,subfig,footnote,graphicx,array,xr,booktabs,placeins}
\usepackage[usenames,dvipsnames]{color}
\usepackage[utf8]{inputenc}
\usepackage[T1]{fontenc}


\theoremstyle{plain}
\newtheorem{lema}{Lemma}
\newtheorem{coro}{Corollary}
\newtheorem{teo}{Theorem}
\newtheorem{prop}{Proposition}
\theoremstyle{definition}
\newtheorem{ap}{Assumption}
\newtheorem{defin}{Definition}
\theoremstyle{remark}
\newtheorem*{rk}{Remark}
\newcommand{\nn}{\mathbf}
\newcommand{\nns}{\boldsymbol}
\newcommand{\Hcal}{\mathcal H}

\newcommand{\fv}[1]{\color{ForestGreen}\textbf{[FV: #1]}\normalcolor}
\newcommand{\lb}[1]{\color{MidnightBlue}\textbf{[LB: #1]}\normalcolor}


\begin{document}
\title[Paleoclimate Reconstruction using INLA.]{Paleoclimate
  Reconstruction and Forecasting of the Northern Hemisphere Temperature using INLA algorithm.}

\author{Luis A. Barboza}
\address{Centro de Investigacion en Matematica Pura y Aplicada (CIMPA), Universidad de Costa Rica\\
San Jos\'e, Costa Rica}
\email{luisalberto.barboza@ucr.ac.cr}


\author{Julien Emile-Geay}
\address{Department of Earth Sciences \\
  University of Southern California \\
  Los Angeles, California, USA.
}
\email{julieneg@usc.edu}

\author{Bo Li}
\address{Department of Statistics \\
  University of Illinois at Urbana-Champaign \\
  Champaign, Illinois, USA.
}
\email{libo@illinois.edu}

\date{\today}
%\date{December 20, 2014}
\keywords{fractional Brownian Motion, Ornstein Uhlenbeck process, Method of Moments}
\subjclass[2010]{62M09, 62F10, 62F12}
\maketitle

\begin{abstract}

\end{abstract}

\section{Introduction.}
\label{sec:intro}

\section{INLA Approach.}
\label{sec:inla}

\section{Datasets.}
\label{sec:data}

\section{Hierarchical Bayesian Model.}
\label{sec:model}

\section{Results.}
\label{sec:results}

\section{Conclusions.}
\label{sec:conclusions}



\end{document}
